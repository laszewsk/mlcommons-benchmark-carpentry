\section{Towards AI benchmark democratizing}
\label{sec:dem}

We have the goal to make AI benchmarking transparent, reproducible, and community-driven. Democratization empowers a broader range of participants to contribute to and learn from AI performance evaluation.

To improve democratization tools, datasets, and evaluation frameworks are openly accessible and easy to use, allowing anyone—from students to independent researchers—to measure, compare, and improve AI models. 

One of the biggest hurdles we find is that some benchmarks, probably rightfully so, target hyperspcale or leadership class machines. However in order o increase the community and raise awareness, smaller scale benchmarks need to be available.

Here is a list of aspects that improve democratization:

\begin{itemize}
    \item \textbf{Accessibility}
    \begin{itemize}
        \item Benchmarks, datasets, and tooling ought to be open-source or freely available.
        \item Users may not need to rely on expensive hardware or proprietary software to participate.
        \item Examples can be leveraged to develop new benchmarks. One can start with examples provided by MLCommons open datasets, pre-built benchmarking pipelines, Jupyter notebooks with ready-to-run benchmarks.
    \end{itemize}

    \item \textbf{Usability}
    \begin{itemize}
        \item Interfaces, documentation, and examples in existing efforts can severe as starting point to developing user-friendly, allowing non-experts to run benchmarks.
        \item Providing automated scripts and tutorials reduce the barrier to entry.
    \end{itemize}

    \item \textbf{Transparency}
    \begin{itemize}
        \item Specifying clear definitions of metrics, scoring methods, and evaluation procedures ensure everyone understands the results.
        \item Improved transparency addresses the hide everything in a “black box” approach,  where only insiders can interpret outcomes.
    \end{itemize}

    \item \textbf{Community Participation}
    \begin{itemize}
        \item Anyone with minimal but sufficient knowledge ought to contribute to benchmarks, improve tools, or submit models.
        \item Democratization also means to encourage collaboration and reproducibility across institutions and geographies, e.g. engaging the community 
    \end{itemize}

    \item \textbf{Impact}
    \begin{itemize}
        \item Through democratization smaller teams or educational institutions can contribute and benefit from learning, competing, and comparing AI benchmarks.
        \item Through democratization fairness and innovation is fostered because knowledge and evaluation methods are dessiminated.
    \end{itemize}
\end{itemize}


