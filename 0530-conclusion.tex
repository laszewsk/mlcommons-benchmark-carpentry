\section{Conclusion}
\label{sec:conclusion}

\TODO{we need to write conclusion}

This comprehensive paper has looked into identifying the motivation and pathway for creating a holistic benchmark  carpentry effort that also focuses on identifying aspects of democratizing AI benchmarks.

This is done by introducing a formal definition of benchmarks that will help to describe them in some uniform fashion, but also 
by identifying a representative set of benchmarks related to AI activities.

Finally,we propose an AI Benchmark Carpentry curriculum that integrates the various topics discussed in  into a structured learning activities 
to empower  
practitioners with reproducible coding practices, experiment‑management 
skills, and an ethical lens on benchmarking. By embedding FAIR principles, 
bias‑mitigation techniques, and performance‑tuning modules, the curriculum 
offers a scalable pathway for communities—from academic labs to industry 
R\&D—to build, share, and improve benchmarks in a collaborative, 
transparent manner.

Together, these activities foster democratization of AI benchmarks and can be utilized to grow the community and the understanding on how benchmarks may effect an individual activity or even community. While deploying such activities we hope to grow community awareness and overcome the lack of well defined activities to educate the community in this regard.
