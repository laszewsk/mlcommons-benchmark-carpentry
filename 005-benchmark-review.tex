\section{Existing AI Benchmarks}
\label{sec:benchmarks}

\TODO{


As part of this we have seen a focus on the creation of machine learning focused benchmarks. According to ChatGPT such benchmarks evaluate ML models, systems, and hardware while using standardized tests and metrics, allowing researchers and engineers to assess performance accurately, compare across models or platforms, and identify areas for improvement.

}

\TODO{
Moved: Most recently, we have seen a number of benchmarking efforts arise that focus on AI-specific tasks. When asking ChatGPT this includes  (a) HPC AI500\TODO{\cite{???}} / HPC AI500 v2\TODO{\cite{???}}: which measures deep learning tasks (e.g. ResNet, Faster RCNN) at scale, focusing on both accuracy and throughput \cite{milabench,www-aibench} (b) MLPerf HPC, which evaluates scientific ML on HPC, measuring memory, I/O, compute, and communication interplay \cite{mlperf}. This even includes community-based benchmarks that address scientific problems bounty incentivized to provide the most accurate solution for a particular problem, such as popularized by Kaggle. \cite{www-kaggle}

}

It is interesting to analyze related work being done in this area, so we queried arXiv~\cite{www-arXiv} and Google Scholar~\cite{www-google-scholar}. Note that Google Scholar does not include all entries from arXiv, but it does include most of them.
As of Oct 1, 2025, we find 106 entries on arXiv when searching for the topic {\em ``AI benchmark''}.
When searching for {\em ``AI and benchmark''}, we find 5,128 entries. Executing the same queries in Google Scholar yields 2,490 entries for {\em ``AI benchmark''} and
2.79 million entries for {\em ``AI and benchmark''}.
We have not yet conducted a deep analysis of this data to categorize the entries or to identify why Google Scholar returns 2.79 million results when both terms are used in conjunction.
It is evident from this that a complete survey is difficult to achieve through manual inspection. In an upcoming effort, we plan to explore how to automatically categorize these entries using LLMs.
To initiate this effort, we issued queries to ChatGPT and obtained an initial list of benchmarks, as shown in Table \ref{tab:scientific_ai_benchmarks}. In addition, we identified several benchmarks based on the interests of members of the MLCommons Science and HPC Working Groups. This includes a categorization of MLCommons benchmarks according to working group interests. \TODO{This has to be updated}

A selected number of these benchmarks, which we reviewed manually, are summarized in more detail on our website at~\cite{www-mlcommons-benchmarks}. A report is currently in preparation at~\cite{www-mlcommons-science-benchmarks-paper}. \TODO{Ben: Gregor will discuss with Ben how we get a one page summary of this with pointer to larger body of work}

\begin{table*}[!ht]
\caption{Scientific AI Benchmarks as found by ChatGPT 4.0 \TODO{evaluate and fix citations, write section about them. Update this table as by now ChatGPT has been updated and may result in mor entries. However these are the entries we used on the Web page.}}
\label{tab:scientific_ai_benchmarks}

\centering
\resizebox{\textwidth}{!}{%
\begin{tabular}{|p{0.12\textwidth}p{0.02\textwidth}|p{0.10\textwidth}|p{0.15\textwidth}|p{0.13\textwidth}|p{0.18\textwidth}|p{0.17\textwidth}|}
\hline
\textbf{Benchmark} &  & \textbf{Scientific Domain} & \textbf{Focus} & \textbf{Task Types} & \textbf{AI Capability Measured} & \textbf{Notable Models Evaluated} \\
\hline
\hline
PDEBench &~\cite{takamoto2022pdebench} & Physics / Engineering & Solving Partial Differential Equations & Time-series simulation, regression & Physics modeling, operator learning & PINNs, FNO, DeepONet \\
\hline
LAB-Bench &~\cite{laurent2024labbench} & Biology / Life Sciences & Laboratory research assistance & Multiple choice, planning, analysis & Scientific reasoning, experiment planning & GPT-4, Claude, Gemini, etc. \\
\hline
SciEval &~\cite{suneval2024} & General Science (multi-domain) & Scientific research understanding & QA, reasoning, synthesis & Recall, reasoning, synthesis, analysis & GPT-4, Claude, Mixtral \\
\hline
SciSafeEval &~\cite{li2024scisafeeval} & Chemistry, Biology, Physics, Medicine & Scientific safety alignment & QA, reasoning, synthesis, prediction & Safety alignment, refusal awareness, adversarial robustness & GPT-4o, Claude, QWen, LLaMa \\
\hline
CORE-Bench &~\cite{siegel2024corebench} & General (CS, Social Sci, Medicine) & Computational reproducibility & Execution, interpretation, synthesis & Reproducibility, multi-modal understanding & GPT-4V, Claude, OpenAI Agents \\
\hline
Galactica Eval &~\cite{taylor2022galactica} & General Science & Scientific writing \& knowledge recall & QA, document generation & Scientific knowledge representation & Galactica, GPT, T5, BERT \\
\hline
Hendrycks Science QA &~\cite{hendrycks2021measuring} & Physics, Chemistry, Biology & High-school level scientific QA & Multiple choice QA & Factual and conceptual understanding & GPT-3, GPT-4, PaLM \\
\hline
ARC &~\cite{clark2018arc} & General Science (K–12) & Commonsense + Science QA & Multiple choice QA & Reasoning, reading comprehension & GPT, T5, UnifiedQA \\
\hline
AristoBench &~\cite{clark2016combining} & Science (elementary–high school) & Natural science exams (NY Regents) & Multiple choice QA & Domain-specific factual recall & Aristo, GPT-3, RoBERTa \\
\hline
SCIQ &~\cite{welbl2017crowdsourcing} & Science (school-level) & Sentence-level inference & MCQ, sentence completion & Language understanding in science context & GPT-3, T5, BERT \\
\hline
BioASQ &~\cite{krithara2023bioasq} & Biomedical / Life Sciences & Biomedical question answering & Factoid, list, yes/no, documents & Domain-specific retrieval and understanding & BioBERT, PubMedGPT, GPT-4 \\
\hline
MedQA (USMLE) &~\cite{jin2021disease} & Medicine & US medical licensing exam QA & Multiple choice, long-form QA & Clinical reasoning, medical knowledge & GPT-4, PaLM, Med-PaLM \\
\hline
MMLU (Science subset) &~\cite{hendrycks2021measuring} & General Science, Math & Academic knowledge & Multiple choice & Cross-domain reasoning and recall & GPT-4, Claude, LLaMA, Gemini \\
\hline
\end{tabular}%
e}
\end{table*}

%\clearpage % Start a new page for the landscape table
%\onecolumn

\begin{landscape} % Rotates the following content to landscape orientation
%\begin{center}
\setlength\LTleft{-2in}     % Align longtable to left margin
\setlength\LTright{-2in}  
\onecolumn

\begin{longtable}{|p{3.5cm}|p{5.5cm}|p{4cm}|p{1.8cm}|p{3.2cm}|p{2.7cm}|}
\caption{Additinal Benchmarks found with ChatGPT}
\label{tab:science-longtable-from-chatgpt}
\\
\hline
\textbf{Benchmark Name} & \textbf{Description} & \textbf{Key Tasks/Evaluation} & \textbf{Metrics} & \textbf{Notable Models/Performance} & \textbf{Resources/Links} \\
\hline
\endhead
\hline
\multicolumn{6}{|r|}{{Continued on next page}} \\
\hline
\endfoot
\hline
\endlastfoot

%\noalign{\smallskip} % Add a small vertical space before the next discipline

% General Scientific Reasoning & Knowledge
\rowcolor{gray!20}
\multicolumn{6}{|l|}{\textbf{General Scientific Reasoning \& Knowledge}} \\
\hline
\textbf{MMLU (Massive Multitask Language Understanding)} & A broad benchmark testing knowledge and reasoning across 57 academic subjects, including various scientific disciplines. & Multiple-choice questions spanning STEM, humanities, social sciences. & Accuracy & GPT-4o, Gemini 1.5 Pro, o1, DeepSeek-R1 (Top models show significant gains) & \href{https://paperswithcode.com/dataset/mmlu}{MMLU} \\
\hline
\textbf{GPQA Diamond} & A challenging subset of the Graduate-Level Google Proof-of-Concept Questions, requiring deep scientific reasoning and knowledge. & Answering difficult, high-level scientific questions, often requiring multi-step reasoning. & Accuracy & o1, DeepSeek-R1 (AI performance has seen remarkable improvements) & 
\TODO{\href{https://gpqa.github.io/}{GPQA}} \\
\hline
\textbf{ARC-Challenge (Advanced Reasoning Challenge)} & Focuses on hard science questions, testing deep understanding and logical reasoning. & Answering science questions, often requiring common sense and deductive reasoning. & Accuracy & (Performance varies, but pushes limits of scientific knowledge) & \href{https://allenai.org/data/arc}{ARC-Challenge} \\
\hline
\textbf{Humanity's Last Exam} & A rigorous academic test designed to push the limits of AI systems in broad academic knowledge. & Answering extremely challenging questions across various fields, including science. & Accuracy & Top systems score low, highlighting remaining challenges for AI. & (Mentioned in AI Index Report) \\
\hline
\textbf{FrontierMath} & A complex mathematics benchmark for evaluating advanced mathematical reasoning in AI. & Solving challenging mathematical problems. & Accuracy & AI systems currently solve a very small percentage of problems. & (Mentioned in AI Index Report) \\
\hline
\textbf{SciCode} & Evaluates coding abilities, specifically relevant to scientific programming and problem-solving. & Solving coding problems related to scientific tasks. & Success Rate & (Part of Artificial Analysis Intelligence Index) & 
\TODO{
\href{https://www.livecodebench.com/}{SciCode (part of LiveCodeBench)}} \\
\hline
\textbf{AIME (American Invitational Mathematics Examination)} & Tests advanced mathematical problem-solving skills, often seen as a proxy for scientific reasoning. & Solving complex math problems. & Accuracy & (Part of Artificial Analysis Intelligence Index) & \href{https://artofproblemsolving.com/wiki/index.php/AIME_Problems_and_Solutions}{AIME} \\
\hline
\textbf{MATH-500} & A mathematics benchmark designed to evaluate models' performance on a diverse set of math problems. & Solving a wide range of math problems. & Accuracy & (Part of Artificial Analysis Intelligence Index) & (Specific link not readily available, often bundled in broader evaluations) \\
\hline
\textbf{CURIE (Scientific Long-Context Understanding, Reasoning and Information Extraction)} & Evaluates LLMs on scientific problem-solving across six disciplines: materials science, condensed matter physics, quantum computing, geospatial analysis, biodiversity, and proteins. & Information extraction, reasoning, concept tracking, aggregation, algebraic manipulation, multimodal understanding, cross-domain expertise from full-length scientific papers. & Task-specific metrics (e.g., accuracy for extraction, correctness for reasoning) & Substantial room for improvement across all models and tasks. & \href{https://arxiv.org/abs/2404.02029}{CURIE Paper} \\
\hline
\textbf{FEABench (Finite Element Analysis Benchmark)} & Measures the ability of LLM agents to simulate, reason, and solve physics, mathematics, and engineering problems using finite element analysis (FEA). & Solving problems requiring FEA simulations and reasoning. & Correctness of solutions & (Part of CURIE related work) & \href{https://arxiv.org/abs/2404.02029}{FEABench Paper} \cite{zhu2024enhancingportfoliooptimizationtransformergan}\\
\hline
\textbf{SPIQA (Scientific Paper Image Question Answering)} & Evaluates multimodal and long-context capabilities of LLMs for scientific image understanding and reasoning. & Answering questions based on scientific figures and associated text from research articles. & Accuracy, F1 score & Significant performance gains by fine-tuning open-source systems. & \href{https://arxiv.org/abs/2404.02029}{SPIQA Paper} \\
\hline % End of General Scientific Reasoning & Knowledge

% \noalign{\smallskip} % Add a small vertical space before the next discipline

% Biomedical & Life Sciences
\rowcolor{gray!20}
\multicolumn{6}{|l|}{\textbf{Biomedical and Life Sciences}} \\
\hline
\hline
\textbf{MedQA} & Assesses clinical knowledge and medical reasoning. & Answering medical questions, often in multiple-choice format. & Accuracy & o1 (OpenAI) has set a new state-of-the-art. & 
\TODO{
\href{https://github.com/pubmedqa/MedQA}{MedQA}} \\
\hline
\hline
\textbf{Therapeutic Data Commons (TDC)} & A foundation for AI in therapeutic science, offering AI-ready datasets, tasks, and leaderboards for drug discovery. & Tasks span drug discovery, including target discovery, activity modeling, efficacy and safety. & Task-specific metrics (e.g., AUC, F1-score) & Wide range of AI models are evaluated. & \href{https://tdcommons.ai/}{TDC} \\
\hline
\textbf{BaisBench (Biological AI Scientist Benchmark)} & Designed to assess AI scientists' ability to generate biological discoveries through data analysis and reasoning with external knowledge in omics data-driven research. & Cell type annotation on single-cell datasets, scientific discovery through multiple-choice questions derived from biological insights of recent single-cell studies. & Accuracy & Current models substantially underperform human experts. & \href{https://arxiv.org/abs/2505.08341}{BaisBench} \\
\hline
\textbf{MOLGEN} & A benchmark for molecular generation tasks in chemistry and drug discovery. & Generating novel molecules with desired properties. & Diversity, validity, novelty, property scores. & (Various generative AI models are benchmarked) & (Often part of broader computational chemistry AI platforms) \\
\hline
\textbf{Open Graph Benchmark (OGB) - Biology} & Graph machine learning benchmarks for biological networks (e.g., protein-protein interaction networks, drug-target interactions). & Node classification, link prediction, graph classification on biological graphs. & Accuracy, AUC & (Various graph neural networks and other graph ML models) & \href{https://ogb.stanford.edu/docs/home/}{OGB} \\
\hline % End of Biomedical & Life Sciences

% \noalign{\bigskip} % Add a small vertical space before the next discipline

% Materials Science & Chemistry
\rowcolor{gray!20}
\multicolumn{6}{|l|}{\textbf{Materials Science \& Chemistry}} \\
\hline
\textbf{Materials Project} & A vast database of material properties, often used as a benchmark for AI models predicting new materials. & Predicting material properties (e.g., bandgaps, formation energies, stability) for inorganic compounds. & Mean Absolute Error (MAE), R-squared & MatGL, ALIGNN, MEGNet, etc. & \href{https://materialsproject.org/}{Materials Project} \\
\hline
\textbf{OCP (Open Catalyst Project)} & Benchmarks for discovering new catalysts using AI, focusing on predicting adsorption energies and forces. & Predicting catalyst properties and reaction outcomes. & MAE on energies and forces & (Various graph neural networks and physics-informed models) & \href{https://opencatalystproject.org/}{Open Catalyst Project} \\
\hline
\textbf{JARVIS-Leaderboard (Joint Automated Repository for Various Integrated Simulations)} & NIST-maintained leaderboards for AI models in materials science, covering various properties and simulations. & Material property prediction (e.g., superconducting transition temperature), image classification in STEM, force field prediction. & Task-specific metrics (e.g., MAE, accuracy) & (Various models for materials design) & \href{https://pages.nist.gov/jarvis_leaderboard/}{JARVIS-Leaderboard} \cite{jarvis}\\
\hline % End of Materials Science & Chemistry
% \noalign{\bigskip} % Add a small vertical space before the next discipline

% Physics & Engineering
\rowcolor{gray!20}
\multicolumn{6}{|l|}{\textbf{Physics \& Engineering}} \\
\hline
\textbf{Quantum Computing Benchmarks (e.g., QML Benchmarks)} & Evaluates AI models for tasks in quantum computing, such as quantum state preparation, quantum control, and error correction. & Optimizing quantum circuits, classifying quantum states. & Fidelity, success probability & (Emerging field with specialized benchmarks) & (Often found in quantum machine learning research papers) \\
\hline
\textbf{Fluid Dynamics Benchmarks (e.g., based on CFD data)} & Benchmarks for AI models in computational fluid dynamics (CFD), such as predicting flow patterns or turbulence. & Solving Navier-Stokes equations, predicting aerodynamic forces. & Error metrics (e.g., L2 error) & (Physics-informed neural networks, traditional ML models) & (Often custom datasets and benchmarks from research labs) \\
\hline % End of Physics & Engineering
% \noalign{\bigskip} % Add a small vertical space before the next discipline

% Earth & Space Sciences
\rowcolor{gray!20}
\multicolumn{6}{|l|}{\textbf{Earth and Space Sciences}} \\
\hline
\textbf{SatImgNet} & Benchmark for analyzing satellite imagery, relevant for climate science, disaster monitoring, and urban planning. & Object detection, semantic segmentation, change detection in satellite images. & mAP, IoU, accuracy & (Various computer vision models) & (Specific datasets and challenges often found in remote sensing conferences) \\
\hline
\textbf{Climate Model Benchmarks (e.g., forecasting)} & Evaluating AI models for climate forecasting, predicting weather patterns, and understanding climate change. & Predicting temperature, precipitation, extreme weather events. & RMSE, bias & (Neural weather models, climate models) & (Collaborative efforts in meteorological and climate science communities) \\
\hline % End of Earth & Space Sciences
% \noalign{\bigskip} % Add a small vertical space before the next discipline

% Cross-Disciplinary/General Purpose AI Benchmarks with Scientific Relevance
\rowcolor{gray!20}
\multicolumn{6}{|l|}{\textbf{Cross-Disciplinary/General Purpose AI Benchmarks with Scientific Relevance}} \\
\hline
\textbf{BIG-Bench (Beyond the Imitation Game Benchmark)} & A very large and diverse benchmark that includes many tasks requiring scientific reasoning and knowledge, often pushing the limits of language models. & Problem-solving, knowledge recall, common-sense reasoning across a wide array of topics, including scientific ones. & Accuracy, various task-specific metrics. & (Many large language models are evaluated on BIG-Bench) & \href{https://github.com/google/BIG-bench}{BIG-Bench} \cite{srivastava2023beyond}\\
\hline
\textbf{CommonSenseQA} & Tests common sense reasoning, which is crucial for scientific understanding and problem-solving. & Answering questions requiring general world knowledge. & Accuracy & (Evaluated across various LLMs) & \TODO{\href{https://www.cs.cmu.edu/~zhiliny/data/commonsense_qa.html}{CommonSenseQA}} \\
\hline
\textbf{Winogrande} & Assesses commonsense reasoning by resolving ambiguities in sentences that require an understanding of context. & Disambiguating sentences based on contextual understanding. & Accuracy & (Evaluated across various LLMs) & \href{https://leaderboard.allenai.org/winogrande/submissions/public}{Winogrande} \cite{winogrande}\\
\hline
\end{longtable}
%\end{center}


\end{landscape}

\twocolumn  %%




\begin{comment}
scimlbench
please add others
Geoffrey pointed many out but i did not have time to write them down

adaptive scale of benchmarks
    x single server
    y single gpu
    run on multiple x and y

impact of network
impact of multiuser

This is translated with Claude via AI from the image:

Screenshot 20250520 at 8.35.23 PM.png
given can you convert the table to a latex table

I'd be happy to convert this table to LaTeX format. Here's the complete LaTeX code for the table:

\end{comment}

ML Benchmarks LaTeX Table \ref{tab:ml_benchmarks}

\begin{table*}[htbp]
\centering
\caption{Machine Learning Benchmarks Across Various Domains \TODO{citations missing}}
\label{tab:ml_benchmarks}
\resizebox{\textwidth}{!}{
\begin{tabular}{|ll|l||l|l|}
\hline
\textbf{Benchmark} & & \textbf{Domain} & \textbf{ML motif} & \textbf{Computing motif} \\
\hline
\hline
jet classification & & HEP/NP & Classification & Embedded \\
\hline
qubit readout & & QIS & Classification, Time-series & Embedded \\
\hline
supernova pointing & & HEP/NP & Regression & Edge Inference \\
\hline
quench detection & & HEP/NP & Classification, Time-series & Embedded, Edge Inference \\
\hline
BraggNN & & Mat Sci & Regression & Embedded \\
\hline
PtychoNN & & Mat Sci & Regression & Embedded, Edge Inference \\
\hline
smart pixels & & HEP/NP & Classification & Embedded \\
\hline
ECON autoencoder & & HEP/NP & Compression & Embedded \\
\hline
MHD regression & & Fusion & Regression & Embedded \\
\hline
Calo challenge & & HEP/NP & Generative & HPC training, Inference \\
\hline
TPBench & & HEP/NP & LLM & HPC training, Inference \\
\hline
Diffuse Multiple Scattering & & Mat Sci & Classification, Computer Vision & Inference \\
\hline
Cloud Masking & & Bio/Env Sci & Semantic Segmentation, Computer Vision & Inference \\
\hline
Electron Microscopy Denoising & & Mat Sci & Computer Vision, Regression & Inference \\
\hline
CosmoFlow & & HEP/NP & Computer Vision, Inverse & HPC training, Inference \\
\hline
Deep CAM & & Bio/Env Sci & Semantic Segmentation, Computer Vision & HPC training, Inference \\
\hline
Earthquake forecasting & & Bio/Env Sci & Regression, Time-series & Inference \\
\hline
STEM DL & & Mat Sci & Classification, Computer Vision & Inference \\
\hline
Uno & & Medical & Regression & Inference \\
\hline
PDE Bench & & Math & PINNs & Inference \\
\hline
The Well & & Math & PINNs & Inference \\
\hline
EAIRA Astro & & HEP/NP & LLM & Inference \\
\hline
EAIRA Molecule & & Mat Sci & LLM & Inference \\
\hline
HDR Anomaly Challenge GW & & HEP/NP & Anomaly Detection & Inference \\
\hline
HDR Anomaly Challenge Butterfly & & Bio/Env Sci & Anomaly Detection & Inference \\
\hline
HDR Anomaly Challenge Sea level & & Bio/Env Sci & Anomaly Detection & Inference \\
\hline
\end{tabular}
}
\end{table*}

\subsection{MLCommons}
\label{subsec:benchmarks-mlcommons}

\TODO{Need volunteer that completes a summary about MLCommons and Table~\ref{tab:mlcommons-benchmarks}}

\TODO{describe what MLCommons is and how it is beneficial for this effort}

\begin{table*}[htb]
\caption{MLCommons Benchmark Overview. \TODO{improve this table.}}
\label{tab:mlcommons-benchmarks}
\resizebox{\textwidth}{!}{
\begin{tabular}{|p{0.15\textwidth}|p{0.15\textwidth}|p{0.15\textwidth}|p{0.15\textwidth}|p{0.15\textwidth}|p{0.2\textwidth}|p{0.3\textwidth}|}
\hline
\textbf{MLPerf Benchmark Name} & \textbf{Task Type} & \textbf{Application Area} & \textbf{Model Types} & \textbf{Metrics} & \textbf{Supported Platforms} & \textbf{Benchmark Features} \\
\hline
\hline
Training & Training (AI models) & AI research, deep learning & CNNs, RNNs, Transformers, etc. & Throughput (samples/sec), Latency (ms) & GPUs (NVIDIA, AMD), TPUs, CPUs, Custom accelerators & Measures training performance for deep learning models \\
\hline
Inference & Inference (AI models) & AI, machine learning, real-time inference & CNNs, Transformers, BERT, etc. & Throughput (samples/sec), Latency (ms) & GPUs (NVIDIA, AMD), TPUs, CPUs, edge devices & Focuses on real-time inference for vision, NLP, etc. \\
\hline
Tiny & Inference (Edge devices) & Edge AI, IoT & CNNs, Transformers, DNNs & Throughput (samples/sec), Latency (ms) & Edge devices (Raspberry Pi, mobile devices, etc.) & Inference on low-power devices, small models \\
\hline
HPC & High-Performance Computing & Scientific computing, AI, supercomputing & DNNs, RNNs, HPC workloads & Time-to-solution, Throughput (samples/sec) & Supercomputers, HPC clusters & High-performance computing with AI applications \\
\hline
Supercomputing/HPC & Training (AI models) & AI, supercomputing & Large-scale DNNs, GPT, etc. & Throughput, Time-to-solution & Supercomputers, multi-node setups & Large-scale training with distributed computing \\
\hline
Vision & Inference (Computer Vision) & Computer vision, autonomous driving, surveillance & CNNs, YOLO, ResNet & Throughput (fps), Latency (ms) & GPUs, TPUs, CPUs, edge devices & Focus on visual processing and autonomous systems \\
\hline
Natural Language Processing (NLP) & Inference (Text processing) & Natural Language Processing, AI chatbots & Transformers, BERT, GPT & Throughput (samples/sec), Latency (ms) & GPUs, TPUs, CPUs, mobile devices & NLP-focused models like BERT, GPT, etc. \\
\hline
Audio & Inference (Audio processing) & Speech recognition, Audio processing & RNNs, CNNs, Transformers & Latency (ms), Throughput & GPUs, TPUs, CPUs, mobile devices & Optimized for speech and audio processing \\
\hline
Data-Centric & Training (AI models) & Data-driven AI tasks & CNNs, RNNs, Transformers, etc. & Time-to-solution, Accuracy, Throughput & GPUs (NVIDIA, AMD), TPUs, custom accelerators & Focus on data-centric machine learning tasks \\
\hline
OpenAI & Training (AI models) & OpenAI-style generative models & GPT, BERT, Transformer-based models & Training speed, Time-to-solution & GPUs (NVIDIA), TPUs, CPUs & Focus on OpenAI-style models like GPT and BERT \\
\hline
\TODO{Science} & Scientific application accuracy & Eartchquake, Cloudmask, others & Time series, ??? & Scientific accuraccy & Supported platforms & depends on application (NVIDIA, AMD) \\
\hline
\end{tabular}
}
\end{table*}

\subsubsection{Benchmarking and in different programming languages}
\label{subsubsec:benchmarks-mlcommons-langs}

\TODO{

Victor could lead:

As you are interested in language and AI benchmark

we may want to add one column or less on  programming language impact

This must include academic references in bibtex format.

 
Please provide summary 

On comparing AI efforts in different programming languages and their impact on benchmarks.

Also identify if the use of highly optimized libraries from vendors have an impact when used in different langauges.

}

\TODO{Describe how different languages need to be addressed. Analyze the benchmark review and find out in which languages they are done. Add that in the table.}

Python
C
Java
Zero-cost abstractions in C++
Julia

Overhead cost discussion

    stopwatch 
    microtimers

\subsubsection{Logging}
\label{subsubsec:benchmarks-mlcommons-logging}

\TODO{describe what logging tools exists for benchmarking}

MLLog
StopWatch
logfiles

\subsubsection{Benchmarking on Desktops}
\label{subsubsec:benchmarks-mlcommons-desktop}

\TODO{ Describe why we not only need HPC benchmarks}

baremetal
containers

\subsubsection{Benchmarking on HPC machines}
\label{subsubsec:benchmarks-mlcommons-hpc}

\TODO{describe why HPC benchmarks are important and why they are difficult. Describe replication need.}

baremetal
containers
Communication: latency, bandwidth, cross-section b/w
Storage: parallel file systems, pre-load buffers and NVMe, no-storage nodes

\subsection{Existing Monitoring Tools in DL and AI}
\label{subsec:benchmarks-monitor}

Logs can be analyzed by streaming SQL engine (e.g., KSQL or Flink SQL) or batch process (e.g., Spark, Hadoop or Hive cluster).
Tensorboard can visualize metrics on TensorFlow, Jax or PyTorch.

What is in TensorFlow? tensor board ...

PyTorch

Alongside active development, the PyTorch team has developed several state of the art monitoring and profiling systems.
The most prominent is the PyTorch Profiles which provides detailed performance metrics, GPU utilization tracking and operator efficiency. It also integrates with TensorBoard for model training and visualization.
There is also the torch.monitor module, that provides a better system logging infrastructure to understand model diagnostics.

There are also paired tools that involve both diagnostics and debugging. Cockpit~\cite{DBLP:journals/corr/abs-2102-06604}, provides real-time monitoring of gradient distributions and curvature, aiding in hyperparameter tuning.
All of these tools come equipped with native benchmarking through TorchBench~\cite{zhao2024deepcontextcontextawarecrossplatformcrossframework}. 
There is also the torch-analyzer for layer-wise runtime and memory mapping statistics.
Holistic profiling solutions have also been developed.
DeepContext links program contexts across execution levels to provide fine-grained performance insights~\cite{zhao2024deepcontextcontextawarecrossplatformcrossframework}.

{happy to keep adding more details on visualizations and evaluations for tf and torch. If not one has claimed it - KM}

others

what many experiment monitoring tools exist, i remember that some companies possibly open source tools offer the collection of multiple results so they can be compared to find best models. Which are they?
    
\subsection{HPC Benchmarking}
\label{subsec:benchmarks-hpc}

\TODO{Piotr}

\subsubsection{TOP500}
\label{subsec:benchmarks-hpc-top500}

The list of world's largest supercomputers has been released biannually
for nearly 4 decades now and thus offers a number of important lessons
in designing benchmarks for longevity, should that ever be the goal of
performance measurements. At the heart of the TOP500 scoring procedure to
arrive at the ranked list of 500 supercomputing installations is the
LINPACK benchmark~\cite{dongarra2003hpl}, which bears the name of the
namesake software library~\cite{dongarra1979linpack} for
solving systems of linear equations. This linear solver package was
designed in the 1970s and implemented in FORTRAN. The user guide for the
library was published in 1979 and had a list only 24
computers~\cite{dongarra1979linpack}. The
following decades brought in various aspects of scaling into the
software, the list sizes, and the machines submitted for inclusion in the
ranking as well as a variety of data and/or reporting information.

\subsubsection{Green500}
\label{subsec:benchmarks-hpc-green500}

It is hard to underestimate the role that power and energy play in the
modern HPC and distributed computing with the multi-mega-watt data
centers and computing facilities abound in many locations across the
globe. But the issues of excessive power draw and/or energy consumptions
data back to the mid-2000s~\cite{feng2005pwrprofsciapps,
cameron2005hpcpowerdistcompsciapps} culminating in a special working
group of cross-industry members~\cite{specpower2008, specpower} and more
importantly by combining the TOP500 ranking with the available power
draw information from the supercomputers culminating in the ranking
called Green500~\cite{feng2007green500}. It is published every since
alongside the TOP500 ranking and it continues to underscore the
importance of efficient use of energy at large HPC installations.

\subsubsection{HPC innovation}
\label{subsec:benchmarks-hpc-innov}

requires monitoring of instantaneous power consumption (levels 1, 2, 3)

\TODO{Piotr: mention something about floating-point and
mixed-precision but I don't want to be taking it in case
someone else already claimed it.}

\subsection{Limitations}
\label{subsec:benchmarks-limits}

\TODO{Tianhao Li} 

\subsubsection{Data Contamination}
\label{subsec:benchmarks-limits-dataContam}

Data contamination occurs when evaluation data overlaps with training data, inflating performance metrics and undermining the validity of benchmarks~\cite{xu2024benchmark}.
This issue arises from the use of large-scale internet-derived corpora in LLM training, leading to unintentional exposure to benchmark data~\cite{xu2024benchmark}. Contamination can take various forms, including exact duplication, syntactic paraphrasing, and exposure to annotation guidelines~\cite{chen2025recent, sainz2023nlp}, resulting in misleading performance evaluations and distorted comparisons between models. While post-hoc detection methods, such as n-gram matching and masked input memorization, help identify contamination, they often fail due to the scale and opacity of pretraining data~\cite{chen2025recent, deng2023investigating, wu2024antileakbench}.
Canary strings, unique, deliberately inserted markers, offer an additional layer of defense by enabling detection of specific data leakage during model training.
By embedding these strings into the benchmark dataset, it becomes possible to track whether a model has memorized benchmark data, alerting researchers to potential contamination when the model generates these canary strings during evaluation~\cite{roberts2023data, ishida2025can}.
Recent proposals for dynamic benchmarking aim to mitigate contamination by generating evaluation data after a model's training period, though challenges remain in standardizing these methods~\cite{chen2025recent, zhu2023dyval, zhu2024dyval, chen2025dynamic}.
Addressing data contamination is crucial for ensuring the reliability and fairness of LLM evaluations.

\subsubsection{Lack of Real-World Relevance}
\label{subsec:benchmarks-limits-relev}

Many widely used evaluations optimize for convenience—clean inputs, single-turn prompts, and static i.i.d.\ test sets—rather than the messy, constraint-laden settings in which models are actually deployed~\cite{liang2023holistic, kiela2021dynabench}.
In production, inputs are noisy, multimodal, and multilingual; tasks are multi-step with tool use (retrieval, code, APIs) and human-in-the-loop escalation; operating conditions impose budgets (latency, compute, and dollar cost), compliance and privacy constraints, and reliability requirements under distribution shift (temporal, geographic, domain)~\cite{koh2021wilds, liang2023holistic}.
Benchmarks that ignore these realities routinely overestimate model utility and under-measure risk~\cite{ribeiro2020beyond, kiela2021dynabench}.
To restore ecological validity, benchmark carpentry should (\romannumeral1) start from ``task stories'' that specify user roles, context, constraints, and success criteria; (\romannumeral2) include end-to-end workflows that require planning, grounding, and tool integration—not just short answers—via interactive, execution-based tasks~\cite{majumdar2025redteamingaired, liu2024agentbench, miehling2025agenticaineedssystems, li2023api, xiong2025butterfly, xie2024osworld}; (\romannumeral3) pair quality metrics with operational ones (latency percentiles, cost per task, failure/abstention rates) and safety metrics (harmful content, data leakage, policy non-compliance)~\cite{gehman2020realtoxicityprompts}; and (\romannumeral4) stress-test robustness with shifted and ``messy'' inputs, long-tail slices, and adversarial/red-team cases~\cite{koh2021wilds, nie2020adversarial, zheng2025all}.
In short, a benchmark should be a faithful proxy for the real job to be done—reflecting workflows, constraints, and risks—not merely a collection of decontextualized questions.

\subsubsection{Reproducibility and Interpretability Challenges}
\label{subsec:benchmarks-limits-reprod}