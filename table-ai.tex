%\clearpage % Start a new page for the landscape table
%\onecolumn

\begin{landscape} % Rotates the following content to landscape orientation
%\begin{center}
\setlength\LTleft{-2in}     % Align longtable to left margin
\setlength\LTright{-2in}  
\onecolumn

\begin{longtable}{|p{3.5cm}|p{5.5cm}|p{4cm}|p{1.8cm}|p{3.2cm}|p{2.7cm}|}
\caption{Additinal Benchmarks found with ChatGPT}
\label{tab:science-longtable-from-chatgpt}
\\
\hline
\textbf{Benchmark Name} & \textbf{Description} & \textbf{Key Tasks/Evaluation} & \textbf{Metrics} & \textbf{Notable Models/Performance} & \textbf{Resources/Links} \\
\hline
\endhead
\hline
\multicolumn{6}{|r|}{{Continued on next page}} \\
\hline
\endfoot
\hline
\endlastfoot

%\noalign{\smallskip} % Add a small vertical space before the next discipline

% General Scientific Reasoning & Knowledge
\rowcolor{gray!20}
\multicolumn{6}{|l|}{\textbf{General Scientific Reasoning \& Knowledge}} \\
\hline
\textbf{MMLU (Massive Multitask Language Understanding)} & A broad benchmark testing knowledge and reasoning across 57 academic subjects, including various scientific disciplines. & Multiple-choice questions spanning STEM, humanities, social sciences. & Accuracy & GPT-4o, Gemini 1.5 Pro, o1, DeepSeek-R1 (Top models show significant gains) & \href{https://paperswithcode.com/dataset/mmlu}{MMLU} \\
\hline
\textbf{GPQA Diamond} & A challenging subset of the Graduate-Level Google Proof-of-Concept Questions, requiring deep scientific reasoning and knowledge. & Answering difficult, high-level scientific questions, often requiring multi-step reasoning. & Accuracy & o1, DeepSeek-R1 (AI performance has seen remarkable improvements) & 
\TODO{\href{https://gpqa.github.io/}{GPQA}} \\
\hline
\textbf{ARC-Challenge (Advanced Reasoning Challenge)} & Focuses on hard science questions, testing deep understanding and logical reasoning. & Answering science questions, often requiring common sense and deductive reasoning. & Accuracy & (Performance varies, but pushes limits of scientific knowledge) & \href{https://allenai.org/data/arc}{ARC-Challenge} \\
\hline
\textbf{Humanity's Last Exam} & A rigorous academic test designed to push the limits of AI systems in broad academic knowledge. & Answering extremely challenging questions across various fields, including science. & Accuracy & Top systems score low, highlighting remaining challenges for AI. & (Mentioned in AI Index Report) \\
\hline
\textbf{FrontierMath} & A complex mathematics benchmark for evaluating advanced mathematical reasoning in AI. & Solving challenging mathematical problems. & Accuracy & AI systems currently solve a very small percentage of problems. & (Mentioned in AI Index Report) \\
\hline
\textbf{SciCode} & Evaluates coding abilities, specifically relevant to scientific programming and problem-solving. & Solving coding problems related to scientific tasks. & Success Rate & (Part of Artificial Analysis Intelligence Index) & 
\TODO{
\href{https://www.livecodebench.com/}{SciCode (part of LiveCodeBench)}} \\
\hline
\textbf{AIME (American Invitational Mathematics Examination)} & Tests advanced mathematical problem-solving skills, often seen as a proxy for scientific reasoning. & Solving complex math problems. & Accuracy & (Part of Artificial Analysis Intelligence Index) & \href{https://artofproblemsolving.com/wiki/index.php/AIME_Problems_and_Solutions}{AIME} \\
\hline
\textbf{MATH-500} & A mathematics benchmark designed to evaluate models' performance on a diverse set of math problems. & Solving a wide range of math problems. & Accuracy & (Part of Artificial Analysis Intelligence Index) & (Specific link not readily available, often bundled in broader evaluations) \\
\hline
\textbf{CURIE (Scientific Long-Context Understanding, Reasoning and Information Extraction)} & Evaluates LLMs on scientific problem-solving across six disciplines: materials science, condensed matter physics, quantum computing, geospatial analysis, biodiversity, and proteins. & Information extraction, reasoning, concept tracking, aggregation, algebraic manipulation, multimodal understanding, cross-domain expertise from full-length scientific papers. & Task-specific metrics (e.g., accuracy for extraction, correctness for reasoning) & Substantial room for improvement across all models and tasks. & \href{https://arxiv.org/abs/2404.02029}{CURIE Paper} \\
\hline
\textbf{FEABench (Finite Element Analysis Benchmark)} & Measures the ability of LLM agents to simulate, reason, and solve physics, mathematics, and engineering problems using finite element analysis (FEA). & Solving problems requiring FEA simulations and reasoning. & Correctness of solutions & (Part of CURIE related work) & \href{https://arxiv.org/abs/2404.02029}{FEABench Paper} \cite{zhu2024enhancingportfoliooptimizationtransformergan}\\
\hline
\textbf{SPIQA (Scientific Paper Image Question Answering)} & Evaluates multimodal and long-context capabilities of LLMs for scientific image understanding and reasoning. & Answering questions based on scientific figures and associated text from research articles. & Accuracy, F1 score & Significant performance gains by fine-tuning open-source systems. & \href{https://arxiv.org/abs/2404.02029}{SPIQA Paper} \\
\hline % End of General Scientific Reasoning & Knowledge

% \noalign{\smallskip} % Add a small vertical space before the next discipline

% Biomedical & Life Sciences
\rowcolor{gray!20}
\multicolumn{6}{|l|}{\textbf{Biomedical and Life Sciences}} \\
\hline
\hline
\textbf{MedQA} & Assesses clinical knowledge and medical reasoning. & Answering medical questions, often in multiple-choice format. & Accuracy & o1 (OpenAI) has set a new state-of-the-art. & 
\TODO{
\href{https://github.com/pubmedqa/MedQA}{MedQA}} \\
\hline
\hline
\textbf{Therapeutic Data Commons (TDC)} & A foundation for AI in therapeutic science, offering AI-ready datasets, tasks, and leaderboards for drug discovery. & Tasks span drug discovery, including target discovery, activity modeling, efficacy and safety. & Task-specific metrics (e.g., AUC, F1-score) & Wide range of AI models are evaluated. & \href{https://tdcommons.ai/}{TDC} \\
\hline
\textbf{BaisBench (Biological AI Scientist Benchmark)} & Designed to assess AI scientists' ability to generate biological discoveries through data analysis and reasoning with external knowledge in omics data-driven research. & Cell type annotation on single-cell datasets, scientific discovery through multiple-choice questions derived from biological insights of recent single-cell studies. & Accuracy & Current models substantially underperform human experts. & \href{https://arxiv.org/abs/2505.08341}{BaisBench} \\
\hline
\textbf{MOLGEN} & A benchmark for molecular generation tasks in chemistry and drug discovery. & Generating novel molecules with desired properties. & Diversity, validity, novelty, property scores. & (Various generative AI models are benchmarked) & (Often part of broader computational chemistry AI platforms) \\
\hline
\textbf{Open Graph Benchmark (OGB) - Biology} & Graph machine learning benchmarks for biological networks (e.g., protein-protein interaction networks, drug-target interactions). & Node classification, link prediction, graph classification on biological graphs. & Accuracy, AUC & (Various graph neural networks and other graph ML models) & \href{https://ogb.stanford.edu/docs/home/}{OGB} \\
\hline % End of Biomedical & Life Sciences

% \noalign{\bigskip} % Add a small vertical space before the next discipline

% Materials Science & Chemistry
\rowcolor{gray!20}
\multicolumn{6}{|l|}{\textbf{Materials Science \& Chemistry}} \\
\hline
\textbf{Materials Project} & A vast database of material properties, often used as a benchmark for AI models predicting new materials. & Predicting material properties (e.g., bandgaps, formation energies, stability) for inorganic compounds. & Mean Absolute Error (MAE), R-squared & MatGL, ALIGNN, MEGNet, etc. & \href{https://materialsproject.org/}{Materials Project} \\
\hline
\textbf{OCP (Open Catalyst Project)} & Benchmarks for discovering new catalysts using AI, focusing on predicting adsorption energies and forces. & Predicting catalyst properties and reaction outcomes. & MAE on energies and forces & (Various graph neural networks and physics-informed models) & \href{https://opencatalystproject.org/}{Open Catalyst Project} \\
\hline
\textbf{JARVIS-Leaderboard (Joint Automated Repository for Various Integrated Simulations)} & NIST-maintained leaderboards for AI models in materials science, covering various properties and simulations. & Material property prediction (e.g., superconducting transition temperature), image classification in STEM, force field prediction. & Task-specific metrics (e.g., MAE, accuracy) & (Various models for materials design) & \href{https://pages.nist.gov/jarvis_leaderboard/}{JARVIS-Leaderboard} \cite{jarvis}\\
\hline % End of Materials Science & Chemistry
% \noalign{\bigskip} % Add a small vertical space before the next discipline

% Physics & Engineering
\rowcolor{gray!20}
\multicolumn{6}{|l|}{\textbf{Physics \& Engineering}} \\
\hline
\textbf{Quantum Computing Benchmarks (e.g., QML Benchmarks)} & Evaluates AI models for tasks in quantum computing, such as quantum state preparation, quantum control, and error correction. & Optimizing quantum circuits, classifying quantum states. & Fidelity, success probability & (Emerging field with specialized benchmarks) & (Often found in quantum machine learning research papers) \\
\hline
\textbf{Fluid Dynamics Benchmarks (e.g., based on CFD data)} & Benchmarks for AI models in computational fluid dynamics (CFD), such as predicting flow patterns or turbulence. & Solving Navier-Stokes equations, predicting aerodynamic forces. & Error metrics (e.g., L2 error) & (Physics-informed neural networks, traditional ML models) & (Often custom datasets and benchmarks from research labs) \\
\hline % End of Physics & Engineering
% \noalign{\bigskip} % Add a small vertical space before the next discipline

% Earth & Space Sciences
\rowcolor{gray!20}
\multicolumn{6}{|l|}{\textbf{Earth and Space Sciences}} \\
\hline
\textbf{SatImgNet} & Benchmark for analyzing satellite imagery, relevant for climate science, disaster monitoring, and urban planning. & Object detection, semantic segmentation, change detection in satellite images. & mAP, IoU, accuracy & (Various computer vision models) & (Specific datasets and challenges often found in remote sensing conferences) \\
\hline
\textbf{Climate Model Benchmarks (e.g., forecasting)} & Evaluating AI models for climate forecasting, predicting weather patterns, and understanding climate change. & Predicting temperature, precipitation, extreme weather events. & RMSE, bias & (Neural weather models, climate models) & (Collaborative efforts in meteorological and climate science communities) \\
\hline % End of Earth & Space Sciences
% \noalign{\bigskip} % Add a small vertical space before the next discipline

% Cross-Disciplinary/General Purpose AI Benchmarks with Scientific Relevance
\rowcolor{gray!20}
\multicolumn{6}{|l|}{\textbf{Cross-Disciplinary/General Purpose AI Benchmarks with Scientific Relevance}} \\
\hline
\textbf{BIG-Bench (Beyond the Imitation Game Benchmark)} & A very large and diverse benchmark that includes many tasks requiring scientific reasoning and knowledge, often pushing the limits of language models. & Problem-solving, knowledge recall, common-sense reasoning across a wide array of topics, including scientific ones. & Accuracy, various task-specific metrics. & (Many large language models are evaluated on BIG-Bench) & \href{https://github.com/google/BIG-bench}{BIG-Bench} \cite{srivastava2023beyond}\\
\hline
\textbf{CommonSenseQA} & Tests common sense reasoning, which is crucial for scientific understanding and problem-solving. & Answering questions requiring general world knowledge. & Accuracy & (Evaluated across various LLMs) & \TODO{\href{https://www.cs.cmu.edu/~zhiliny/data/commonsense_qa.html}{CommonSenseQA}} \\
\hline
\textbf{Winogrande} & Assesses commonsense reasoning by resolving ambiguities in sentences that require an understanding of context. & Disambiguating sentences based on contextual understanding. & Accuracy & (Evaluated across various LLMs) & \href{https://leaderboard.allenai.org/winogrande/submissions/public}{Winogrande} \cite{winogrande}\\
\hline
\end{longtable}
%\end{center}


\end{landscape}

\twocolumn  %%


